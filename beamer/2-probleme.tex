\section{Modélisation du système et du problème}

\subsection[Modèle initial]{Formulation et modélisation initiale du problème}
\begin{frame}{Formulation initiale}
  \begin{itemize}
  \item Minimiser le temps d'exécution de la mission
  \item sous contraintes
    \begin{itemize}
    \item \( \forall k \in \mathcal{K}, P_{k, \text{succ}} \geq
      \bar{P} \),
          \note[item]{\( \mathcal{K} \) ensemble des terminaux }
    \item vitesse max de l'UAV \( V_{\text{max}} \)
    \end{itemize}
  \end{itemize}
      
\end{frame}



\begin{frame}{Modélisation temps et position}
  \begin{block}{Discrétisation}
    \[ T = M \delta_t \]
    \note[item]{\( T \) horizon de la mission}
    \note[item]{\( M \) pas de temps final}
    \note[item]{\( \delta_t \) durée d'un pas de temps}
    \begin{equation}
      \min \text{\textit{temps d'exécution de la mission}}
      \longleftrightarrow \min_M M
    \end{equation}
  \end{block}
  \begin{block}{Position et vitesse}
    \begin{itemize}
    \item Position \( \mathbf{q}[m]_{m = 1}^M \in \mathbb{R}^{2 \times
      M} \)
    \item Déplacement maximal \( \tilde{V}_{\text{max}} = \delta_t
      V_{\text{max}} \)
    \end{itemize}
    \begin{equation}
      \text{\textit{UAV restreint à}} V_{\text{max}} \longleftrightarrow
      \forall m, \| q[m] - q[m - 1] \| \leq \tilde{V}_{\text{max}}
    \end{equation}
  \end{block}
\end{frame}
\begin{frame}{Réception des paquets}
  \begin{itemize}
  \item \( N_k \) nombre de paquets reçus par terminal \( k \)
  \item \( N' \) nombre de paquets nécessaire
  \item \( N \) nombre de paquets codés transmis
    \note[item]{\( N \) donné par RLNC}
  \end{itemize}
  \begin{block}{Qualité de transmission}
    \begin{equation}
      \text{\textit{probabilité de réception correcte}} \longleftrightarrow
      \mathbb{P}(N_k \geq N')
    \end{equation}
    \note[item]{
      avec \( N_k \sim \mathcal{P} \) vu que \( N_k = \sum_{m = 1}^M
      \sum_{l = 1}^L Z_k[m, l] \) avec \( Z_k[m, l] \) suivant une loi de
      Bernouilli indiquant si terminal \( k \) a reçu le paquet \( l \) du
      time slot \( m \)
    }
  \end{block}
\end{frame}
\begin{frame}{Modélisation initiale}
  \begin{align}
      & \min_{q[m]_{m=1}^M, M} M \tag{P1} \label{eq:init-obj} \\
      \text{s.t. } & \forall k \in \mathcal{K}, P_{k, \text{succ}}
      \geq \bar{P} \label{eq:init-prec} \\
      & \forall m \in \llbracket 1, M \rrbracket, \| q[m] - q[m - 1]
      \| \leq \tilde{V}_{\text{max}}
  \end{align}
\end{frame}

\subsection{Minoration de la probabilité de succès}
\begin{frame}{Minoration de \( P_{k, \text{succ}} \)}
  \begin{alertblock}{Problème}
    \( P_{k, \text{succ}} \) dépend de \( \mathbf{q} \)
    \note[item]{probabilité de réception diminue quand la distance
      augmente}
  \end{alertblock}
  \begin{block}{Distance caractéristique}
    On introduit \( D \) tel que
    \[ \| \mathbf{q}[m] - \mathbf{w}_{k} \| > D \implies
      \text{réception impossible} \]
  \end{block}
\end{frame}
\begin{frame}{Minoration}
  \( \mathcal{M}_{k, D} = \{ m | \| \mathbf{q}[m] - \mathbf{w}_k \|
  \leq D \} \)
  \note[item]{\( \mathcal{M}_{k, D} \) ensemble des pas de temps
    durant lesquels la transmission est réalisable}
  \begin{block}{Minorant}
    \( \hat{N}_k \) un minorant du nombre de paquets reçus par
    terminal \( k \)
    \note[item]{\( \hat{N}_k \sim \mathcal{B}(| \mathcal{M}_{k, D} |
      L, p_D)\)}
    \note[item]{\( L \) nombre de paquets transmis par time slot}
    \note[item]{\( p_D \) proba de réception d'un paquet à distance \(
    D \)}
    \begin{equation}
      P_{k, \text{lb}} = \mathbb{P}(\hat{N}_k \geq N')
    \end{equation}
  \end{block}
\end{frame}

\subsection{De la probabilité au temps}
\begin{frame}{De la probabilité au temps}
  \begin{block}{Approximation}
    \[ \hat{N}_k \approx \mathcal{G}(\mu, \sigma^2) \]
    \( \mu, \sigma^2 \) dépendant de \( | \mathcal{M}_{k, D} |, L, p_D
    \)
  \end{block}
  \begin{equation}
    P_{k, \text{lb}} = \mathbb{P}(\hat{N}_k \geq N) \approx
    Q\left(f(N', | \mathcal{M}_{k, D} |, L, p_D) \right)
  \end{equation}
  \note[item]{\( Q \colon x \mapsto \int_x^{\infty} e^{-u^2 / 2} \,
    \mathrm{d}u \)}
  \note[item]{\( f \) est une fonction connue}
  \begin{block}{Expression en temps}
    \begin{equation}\label{eq:1}
      | \mathcal{M}_{k, D} | \geq M_{\text{min}}
    \end{equation}
    \( M_{\text{min}} \) connu dépendant de \( L, p_D, N' \) et \(
    Q^{-1}(\bar{P}) \)
  \end{block}
\end{frame}
\begin{frame}{C'est bientôt la fin}
  \begin{align}
    \label{eq:2}
    & \min_{q[m]_{m=1}^M, M} T = \delta_t M \tag{P3} \\
    \text{s.t. } & \forall k \in \mathcal{K},
                   | \mathcal{M}_{k, D} | \geq M_{\text{min}} \\
    & \forall m \in \llbracket 2, M \rrbracket,
      \| q[m] - q[m - 1] \| \leq \tilde{V}_{\text{max}}
  \end{align}
\end{frame}

\subsection{Trajectoire explicitée}
\begin{frame}{Re modélisation de la trajectoire}
  \begin{block}{Relaxation du TSP}
    Regroupement des terminaux en \( \mathcal{S}_g \)
    \note[item]{\( \mathcal{S}_g \) zone de rayon \( D \) regroupant
      plusieurs terminaux}
    \note[item]{Zones \( \mathcal{S}_{g} \) données par un algo non
      exposé ici}
    \note[item]{Pour couvrir tous les terminaux de la zone \(
      \mathcal{S}_g \), il suffit de passer dans la zone}
  \end{block}
  \begin{align}
    & \min_{
        {\{\mathbf{s}_g, \mathbf{f}_g\}}_{g=1}^G
      }
      \underbrace{
      \sum_{g = 1}^G \max \left\{
      \frac{
      \| \mathbf{f}_g - \mathbf{s}_g \|
      }{
      V_{\text{max}}
      }, T_{\text{min}}
      \right\}
      }_{\text{traversée}} +
      \underbrace{
      \sum_{g = 1}^G
      \frac{
      \| \mathbf{s}_{g + 1} - \mathbf{f}_g \|
      }{
      V_{\text{max}}
      }
      }_{\text{transition}} \tag{P4} \\
    \text{s.t } & \forall g, (\mathbf{s}_g, \mathbf{f}_g) \in \mathcal{C}_g
  \end{align}
\end{frame}

\subsection{Optimisation de la vitesse}
\begin{frame}{Optimisation de la vitesse}
  \begin{block}{Variable de décision}
    \begin{itemize}
    \item Discrétisation en espace, \( \delta_t \rightsquigarrow
      \delta_d \)
    \item Manipulation du temps, \( \mathbf{q}[m] \rightsquigarrow
      \tau_j \)
    \end{itemize}
  \end{block}
  \begin{align}
    & \min_{\{\tau_j\}_{j = 1}^J} \sum_{j = 1}^J \tau_j \tag{P5} \\
    \text{s.t. } & \forall k \in \mathcal{K}, \sum_{j = 1}^J I_{kj}
                   \tau_j \geq T_{\text{min}} \\
    & \forall j \in \llbracket 1, J \rrbracket,
      \tau_j \geq \frac{\delta_d}{V_{\text{max}}}
  \end{align}
  \note[item]{Discrétisation en espace donne \( \{\mathbf{q}_j\}_{j =
  1}^J \) ensemble de positions à chaque pas de déplacement}
  \note[item]{Avec \( I_{kj} = 1 \) si UAV en connexion avec terminal
    \( k \) quand en \( j \)\textsuperscript{ième} position}
\end{frame}
%%% Local Variables:
%%% mode: latex
%%% TeX-master: "atp"
%%% End:
