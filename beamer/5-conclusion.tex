\section{Conclusion}


\begin{frame} {Résumé}

Cet article a donc présenté 
\begin{itemize}
	\item un cas d'optimisation de trajectoire 4D
	\item contraint par la vitesse maximale du drone
	\item et par le temps minimum de connexion nécessaire à bonne réception du message. 
\end{itemize}


\end{frame}


\begin{frame} {}

Pour ce faire, plusieurs étapes ont été nécessaires

\begin{enumerate}
	\item Reformulation du problème en utilisant une seule contrainte
	de temps minimum de connexion entre le drone et le terminal terrestre.\pause
	
	\item Les auteurs ont ensuite montré que la trajectoire optimale
	peut-être constituée uniquement de segments de droites reliant des
	waypoints dont la position est optimisée.\pause
	\item Ils ont calculé la position optimale de ces waypoints.
	Ce qui définit une trajectoire optimale. \pause 
	\asuivre
	
\end{enumerate}
\end{frame}	
	
\begin{frame} {Résumé}



\begin{enumerate}	
	
	 \suite
	\item Puis ils optimisent la vitesse en fonction du temps le long de la trajectoire obtenue
	en utilisant la programmation linéaire(LP).\pause
	\item Les résultats numériques ont mis en évidence des performances significativement améliorées
	 par rapport à une approche heuristique de conception de trajectoires ou un système multicast statique.\pause
	\item Ce qui tend à montrer le grand potentiel des drones de 
	télécommunication à usage de transmetteurs multicast dans les réseaux sans-fil.
	 
\end{enumerate}
\end{frame}



\begin{frame} {Critiques}
\begin{itemize}
	\item Article bien écrit qui couvre un spectre de connaissances assez large : telecom, stat, RO, optimisation.
	\item Article assez difficile car couvre divers domaines : Télécom., Recherche Opérationnelles, Statistiques
	\item Plan annoncé, suivi et rappelé dans la conclusion
	\item Notations bien explicitées
	\item Bibliographie pertinente
	\item Il manquerait une application concrète, in situ, des résultats obtenus.
	\item La notation F est trompeuse. Dans l'article, elle représente la complémentaire de la fonction de répartition, cette dernière étant généralement notée $\tilde{F}$.
\end{itemize}


\end{frame}



\begin{frame} {Perspectives}

\begin{enumerate}
	\item Phase multicast de transmission de paquets
	\item Phase device to device (D2D) : les terminaux s'échangent des paquets pour reconstituer le message dans leur totalité. 

\end{enumerate}



% \begin{figure}[t]
%	\centering
%	\includegraphics[height=\dimexpr11\textheight/16\relax]{d2d}
%	\caption{D2D}
%\end{figure}

\begin{figure}
	\centering
	\includegraphics[width=1\linewidth]{d2d}
%	\caption{}
	\label{fig:d2d}
\end{figure}


\end{frame}


%\begin{frame} {}
%
%Cet article n'a pris en compte que la phase multicast. L'étude conjointe
%des deux phases multicast et D2D serait probablement intéressante à entreprendre.
%En lien avec des techniques de clustering pour les stations sol, l'optimisation
%conjointe pourrait permettre de réduire davantage les coûts de transmission et donc la taille des drones
%utilisés.
%
%\end{frame}