

\section {Introduction}

\begin{frame}{Application des drones aux télécommunications}
\begin{itemize}
\item Les drones trouvent de nombreuses utilisations tant ludiques que professionnelles. 
\item Cet article présente une application dans le domaine des télécommunications dans le cas particulier où le drone vole à faible hauteur, typiquement H = 100m.
\item La mobilité des drones peut permettre d'améliorer les performances qui seraient obtenues pas un système
utilisant des relais ou des stations de base terrestres.
\item Exemple : récolte d'informations énergétiquement efficace dans un contexte IoT du fait de la proximité du drone.
\item Dans ce type d'utilisation, les performances du système dépend fortement des trajectoires du drone qu'il s'agit d'optimiser.
\end{itemize}
  

\end{frame}

\begin{frame}{Contexte de l'article}
\begin{itemize}
	
	\item Date de première publication le 12 Janvier 2018.
	
	\item Publié aussi dans le journal IEEE (Institute of Electrical and Electronics Engineers) Transactions on Wireless Communications 
	( Volume: 17 , Issue: 4 , April 2018 ) qui s'adresse à un public intéressé par les communications sans fils.
	
	\item Article visant un objectif essentiellement concret

	
\end{itemize}
\end{frame}

\begin{frame}{Contexte de l'article}
\begin{itemize}
	\item Financé par l'Université de Singapour.

	\item Bibliographie effectuée : 
	\begin{itemize}
		\item Etat de l'art dans les domaines relatifs aux UAV et aux telecommunications, 
		\item Articles ou manuels plus théoriques, notamment sur la résolution du TSP, publiés par 
 			\begin{itemize}
 				\item  Dans le "European Journal of Operational Research"
 				\item  Dans le "Journal of Algorithms"
 				\item  ou par "Cambridge Univ. Press" (convex optimisation), l'article référence un manuel
 				en optimisation convex, téléchargeable gratuitement et associé à un MOOC de Stanford University.
 			\end{itemize}
	\end{itemize}


\end{itemize}
\end{frame}
 
\begin{frame}{Dissémination multicast sans fil}
\begin{itemize}
	\item Drone  équipé d'un système de transmission  sans fil
	\item K Terminaux terrestres
	\item Objectif : envoi d'un message par paquets via communication sans fil : taille W = 2 Mbits par ex.  
\end{itemize}
%Le présent article étudie un drone déployé en vue de disséminer vers K Terminaux Terrestres (GT : Ground Terminals) un message (e.g. W = 2 Mbits) via une connexion sans fil. On parle de multicast, et non de broadcast, car à un instant donné seule une partie des terminaux (GT) est à portée du drone.
%
%\comment{GH:Ça fait beaucoup de texte non?  Je suggère de diminuer
%  drastiquement le contenu, et de compter sur l'oral, quitte à mettre
%  du contenu en note}


\begin{figure}
	\centering
	\includegraphics[width=0.6\linewidth]{images/multicast}
	\caption{}
	\label{fig:multicast}
\end{figure}

\end{frame}

\begin{frame}{Minimisation du temps de mission}
%Les auteurs cherchent à minimiser le temps de mission tout en assurant une réception du message par les GT avec une probabilité cible fixée à l'avance. Le tout sous la contrainte de vitesse maximale du drone. La probabilité de réception d'un GT est difficile à évaluer car c'est fonction compliquée de la trajectoire du drone. Pour contourner le problème, les auteurs reformulent le problème en introduisant un paramètre auxiliaire D (distance critique horizontale entre le drone et la GT). Au delà de cette distance D, le nombre de paquets perdus devient trop important car le temps de connexion devient insuffisant. 

%\comment{GH: idem, beaucoup, beaucoup de texte\dots}

\begin{itemize}
	\item Objectif : Minimiser le temps de mission total T
	\item En assurant une bonne réception des paquets avec une probabilité suffisante
	\item Sous la contrainte de vitesse max Vmax du drone 
\end{itemize}

\end{frame}


\newcounter{sauvegardeenumi}
\newcommand{\asuivre}{\setcounter{sauvegardeenumi}{\theenumi}}
\newcommand{\suite}{\setcounter{enumi}{\thesauvegardeenumi}}

\begin{frame}{Plan suivi par les auteurs}

\begin{enumerate}
	
	\item Reformulation du problème en utilisant une seule contrainte
	de temps minimum de connexion entre le drone et le terminal terrestre.	
	\item Démonstration du théorème : " la trajectoire optimale
	peut-être constituée uniquement de segments de droites reliant des
	waypoints dont la position est optimisée."

	
\end{enumerate}
\asuivre
\end{frame}

\begin{frame}{Plan suivi par les auteurs}
 
\begin{enumerate}
	\suite
	\item Utilisation du concept de "stations de base virtuelles" (VBS) pour calculer ces waypoints ;  problème qui est en fait un TSP généralisé (NP-Hard).

	\item Optimisation de la vitesse en fonction du temps le long de la trajectoire obtenue
	en utilisant la programmation linéaire (LP).

\end{enumerate}

\end{frame}


\begin{frame}{Plan}
	\tableofcontents
\end{frame}
%%% Local Variables:
%%% mode: latex
%%% TeX-master: "atp"
%%% End:
